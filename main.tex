% !TEX encoding = UTF-8 Unicode
% !TEX program = xelatex
% !BIB program = biber
% !TEX TS-program = xelatex
% !BIB TS-program = biber
%%
%%  本模板方式编译: XeLaTeX + biber
%%
%%  注意: 在改变编译方式前应先删除 *.toc 和 *.aux 文件
%%
\documentclass[12pt,openright]{book}

% 引入NKThesis包
\usepackage[emptydoublepage]{NKThesis}   % 中文
%\usepackage[emptydoublepage,English]{NKThesis} % 英文

% 其它包按需添加
% \usepackage{amsmath}
% \usepackage{cases}
% \usepackage{multirow}
% \usepackage{tocbibind}
\usepackage{ulem}
\usepackage{xeCJKfntef} %CJKunderline
\xeCJKsetup{underline= {format = \color{black}, thickness=0.4pt}} %CJKunderline setting

% 参考文献
\addbibresource{nkthesis.bib}
% 图片文件夹
\graphicspath{{image/}}

\includeonly{
	./tex/abstract,
	./tex/introduction,
	./tex/relatedwork,
	./tex/method,
	./tex/discussion,
	./tex/summary,
	./tex/references,
	./tex/acknowledgements,
	./tex/appendices,
	./tex/resume
}
\begin{document}
%  设置基本信息
%  注意:  逗号`,'是项目分隔符. 如果某一项的值出现逗号, 应放在花括号内, 如 {,}
%
\NKTsetup{
	% 封面设置
	论文题目(中文) = 我是爱南开的,
	论文题目(英文) = I Love Nankai,
	学号           = 19190062,
	姓名       = 周恩来,
	年级          = 1919级,
	学院          = 管理学院,
	系别          = 政治系,
	专业          = 国际政治,
	完成日期      = ,
	指导教师       = 张伯苓教授,
	校外指导教师   = ,
}


% !TeX root = ../main.tex
% -*- coding: utf-8 -*-


\begin{zhaiyao}

这里输入中文摘要。
\end{zhaiyao}




\begin{guanjianci}
毕业论文;模板
\end{guanjianci}



\begin{abstract}


This is the abstract.

\end{abstract}



\begin{keywords}
Thesis; template
\end{keywords} 

\tableofcontents

% !TeX root = ../main.tex
% -*- coding: utf-8 -*-
% !TeX root = ../main.tex
% -*- coding: utf-8 -*-

\chapter{绪论}
\label{chpt:introduction}


广泛流行的社交网络和移动设备使人类能够对他们关于线上购买的商品和服务运用文本,图片和视频等方式评价和分享他们的观点和想法\cite{zhao2015predicting,zhao2016predicting,deriu2017leveraging,gong2017clustered,luiz2018feature,gong2018sentiment,zhao2020discrete,zhao2020end}.例如,当计划购物时,很大概率我们会看其他人对于这个商品或服务的评价.如果用户的反馈是以负面评价主导的,我们也许会转而考虑其他的品牌.对于用户生成的大规模多媒体数据进行情感分析不仅可以帮助顾客选择他们需要的物品并且还可以促使企业提高他们产品和服务的质量\cite{zhao2016predicting,chen2019emoji}.最近的研究\cite{zhang2018textual,kiritchenko2014sentiment,georgakopoulos2018convolutional,liu2018content,yadav2020sentiment,wang2018sentiment,wang2019aspect,chen2019emoji,biddle2020leveraging}已经证明了深度神经网络(DNN)在情感分析中已经取得了最前沿的表现.但是,训练一个深度神经网络并使其达到它的能力上限通常需要很大规模的标注数据,这需要大量的时间和人力资源才能得到.一个方案是在一个有标注数据的源域上面训练并在目标域上直接使用.但是由于域间隙的存在\cite{torralba2011unbiased},\textit{i.e.} 源域和目标域的分布不同,这种方案将会导致神经网络在目标域上的表现明显退化\cite{tzeng2015simultaneous,hoffman2018cycada,yue2019domain,yang2020curriculum}.最小化域间隙的领域自适应(DA)\cite{patel2015visual,sun2015survey,kouw2019review,zhao2020multi,zhao2020review}方法提供了一个替代方案:在源域上训练一个对于目标域泛化能力强的模型.
目前的领域自适应方法主要集中于单源无监督学习\cite{liu2019survey,xi2020domain},\textit{i.e.} 只有一个有标注过的源域和一个无标注的目标域.然而这些无监督领域自适应(UDA)方法当且仅当源域与目标域的域间隙相对较小时才会表现很好,在实际场景下,域间隙很大或源域的个数大于一时,它们的表现会严重退化\cite{guo2018multi,zhao2020multi}.例如,如果我们有一个目标域:厨房,其中包含用户对于碗,菜谱,平底锅和电水壶等评论,以及三个源域:书籍,电器和电影, 完美地将每个源域与目标域对齐是难以实现的.一个直观的解决方案是将三个源域的数据组合在一起成为一个源域,之后通过单源无监督领域自适应的方法对齐源域与目标域.然而这样的方法只会得到次优结果,如图1.充分探索不同源域间的复杂相关信息能够在目标域上得到更好的结果,这催生了多源域适应(MDA)方法\cite{sun2015survey,zhao2020multi}.
最近有一些深度多源域适应方法被提出,其中大部分基于对抗学习,其中包含一对特征提取器和域判别器,用来最小化特征空间中源域分布与目标域分布的JS散度(\textit{e.g.}:MDAN\cite{zhao2018adversarial},MOE\cite{guo2018multi}),还有一部分基于最小化分布差异,其中包含一个特征提取器和一个核函数,用来最小化再生希尔伯特空间中源域和目标域的分布差异损失函数(例如:MMD损失\cite{dziugaite2015training},CDD损失\cite{Kang_2019_CVPR}).这些方法主要集中于抽取不同领域的域无关特征,将每个源域与目标域分别对齐,或从统计意义上对给源域样本设置权重.这些方法虽然可以得到不同领域的域无关特征,但仍然有一些局限.首先,在特征抽取的过程中一些目标域上的与情感相关的域特征被丢失,这是因为域间共享的特征提取器旨在通过将源域和目标域样本投影到一个低维空间中来提取域无关特征,因此不会包含目标域中全部的情感相关特征,同时可以理解为这是一个压缩的不可逆过程,易导致模式崩塌\cite{zhu2017unpaired,zhao2019cycleemotiongan,hoffman2018cycada}.其次,一些现存的多源域适应方法利用域标签将源域和目标域分别对齐,这忽视了不同源域甚至同一领域的不同子域间的相互关系.这些方法在域标签不存在的情况下将会自然地退化为单源域适应并得到次优解.最后,现存的基于采样的方法集中于通过训练样本选择模型来筛选距离目标域较近的源于样本(\textit{e.g.}:MDDA\cite{zhao2020distilling},CMSS\cite{yang2020curriculum}),这并不能即时并准确地在不同的训练阶段量化样本的动态最优权重,这是因为在不同的训练阶段,模型对同一个样本的学习进度是不同的,而一个收敛的,以样本为输入,以权重为输出的样本选择模型需要在不同的训练阶段对同一个样本输入去输出不同的权重,这与收敛的定义相悖.
在这篇文章中,我提出了一个新的实例级多源域适应框架,命名为课程式循环一致性生成对抗网络(C-CycleGAN),来解决上述问题.首先,为了将源域和目标域中样本的全部信息映射到一个连续平滑的表征空间并最小化信息损失,我引入了重构方法以更好地保留信息.其次,对于编码后的源域样本表征,我利用循环一致性生成对抗网络(CycleGAN)生成了一个中间域以将混合后的源域和目标域进行对齐.为了量化一个batch中不同源域样本对于模型的重要性,我通过动态的基于模型和无模型的两种权重机制对实例级样本分配权重.情感损失函数同样会反向传播给源域到目标域的生成器来保留情感信息.我在三个基准数据集上做了大量实验:Reviews-5\cite{yu2016learning},Amazon benchmark\cite{chen2012marginalized},和Multilingual Amazon Reviews Corpus\cite{chen2019multi}.结果表明提出的C-CycleGAN模型超越了情感分类任务领域自适应的SOTA方法.
综上所述, 本片文章的贡献分为三部分:
(1)我提出了一个新的多源域适应方法,命名为课程式循环一致性生成对抗网络(C-CycleGAN)以最小化多个源域与目标域之间的域间隙.
(2)我设计了新的实例级基于模型和无模型两种权重机制,可以动态更新样本权重.利用这种方法,C-CycleGAN不需要任何域标签,并且可以更好地挖掘多源域间的复杂相互关系.实验表明利用这种方法得到的结果优于利用域标签.
(3)我在三个基准数据集上做了大量实验.与最佳的基线模型相比,C-CycleGAN在平均分类准确率这一指标上在Reviews-5,Amazon benchmark和Multilingual Amazon Reviews Corpus数据集上分别提升了1.6\%,1.2\%和13.4\%.
% !TeX root = ../main.tex
% -*- coding: utf-8 -*-


\chapter{相关工作} 
\label{chpt:relatedwork}

\textbf{文本情感分类} 文本情感分类或意见挖掘旨在从文本方面评估人们对于商品,
服务或组织的观点,情感和态度\cite{zhang2018deep}. 诸如商品评论,论坛讨论和微信等广泛流
行的社交网络促成了这一任务的迅速发展\cite{zhang2018deep,chen2019emoji}. 传统的情感分析方法主
要专注于设计人工特征\cite{pang2008opinion,mohammad2013nrc},将其送入标准的分类器,例如SVM. 近年
情感分析的工作主要基于深度神经网络\cite{zhang2018deep,wang2018sentiment},其已经胜任了许多自然语
言处理任务. 一些广泛应用于情感分析的深度模型包括递归自编码器\cite{socher2011semi,dong2014adaptive,qian2015learning},
递归神经张量网络\cite{socher2013recursive},循环神经网络(RNN)\cite{cho2014learning},长短期记忆神经网络(LSTM)\cite{hochreiter1997long},
树状LSTM\cite{tai2015improved},RNN编码解码结构\cite{cho2014learning}和BERT\cite{devlin2019bert}. 上述监督学习方法通常需要大
规模的标注数据进行训练. 然而高质量的情感标签通常需要消耗大量的时间和人力
资源才能得到. 在C-CycleGAN中,我应用了双向LSTM(Bi-LSTM)\cite{hochreiter1997long}作为编码器
和一个多层感知机作为最终情感分类任务的分类器. \\
\textbf{单源无监督域适应} 近年单源无监督域适应(SUDA)方法大多主要采用包含两条尾部相
连的数据流的深度神经网络\cite{zhuo2017deep,zhao2018adversarial}. 一条数据流在带标记数据的源域上用传统的
目标任务损失函数进行训练,例如分类任务的交叉熵损失(cross entropy loss). 
另一条数据流旨在通过不同的对齐损失函数(alignment loss)将源域与目标域进行
对齐以解决域间隙问题,例如分布差异损失(discrepancy loss),对抗损失(adversarial loss)
和自监督损失(self-supervised loss)\emph{etc}. 基于分布差异的方法采用一些距离度
量来最小化源域和目标域在特定空间的差异,例如最大期望差异\cite{long2015learning,wang2018multi,xi2020domain},相
关性对齐\cite{sun2016return,sun2017correlation,zhuo2017deep}和对比领域差异\cite{kang2019contrastive}. 对比判别模型
(adversarial discriminative model)通常采用一个域判别器以对抗地对齐从
源域和目标域中抽取的特征来使他们无法判别\cite{ganin2016domain,tzeng2017adversarial,chen2017no,shen2017wasserstein,tsai2018learning,huang2018domain,kumar2019adversarial,wu2020unsupervised}. 
除了域判别器,对抗生成模型(adversarial generative model)还包含一个生成部分
,基于生成对抗网络(GAN)\cite{goodfellow2014generative}和它的变体,例如CoGAN\cite{liu2016coupled}, SimGAN\cite{shrivastava2017learning}和
CycleGAN\cite{zhu2017unpaired,zhao2019cycleemotiongan,hoffman2018cycada}以生成假源域或目标域样本. 基于自监督的模型在原始的目
标任务网络中融入了辅助的自监督学习任务以使源域和目标域靠近. 通常使用的自监
督任务包括重构(reconstruction)\cite{ghifary2015domain,ghifary2016deep,chen2020fido},图像旋转预测(image rotation prediction)\cite{sun2019unsupervised,xu2019self}
,拼图预测(jigsaw prediction)\cite{carlucci2019domain},和掩码(masking)\cite{vu2020effective}. 尽管这些方法在
SUDA任务中达到了满意的效果,他们的表现在应用于多源域适应任务时会大幅衰减. \\
\textbf{多源域适应} 基于一些理论分析\cite{ben2010theory,hoffman2018algorithms},多源域适应旨在更好地解决训练数据属于
多个源域的问题\cite{sun2015survey,zhao2019multi}. 早期的多源域适应方法主要分为两种\cite{sun2015survey,sun2011two,duan2012exploiting,chattopadhyay2012multisource,duan2012domain}以及预训练分类器的组合\cite{xu2012multi,sun2013bayesian}. 近年的方法考虑了一些特殊的
多源域适应场景,例如不完全多源域适应\cite{ding2018incomplete}和目标域偏移\cite{redko2019optimal}. 
最近一些基于深度表示学习的多源域适应方法被提出,例如多源对抗网络(MDAN)\cite{zhao2018adversarial},
深度混合网络(DCTN)\cite{xu2018deep},混合专家模型(MoE)\cite{guo2018multi},时矩匹配网络(MMN)\cite{peng2019moment},
多源对抗聚合网络(MADAN)\cite{zhao2019multi},多源蒸馏模型(MDDA)\cite{zhao2020distilling}和课程式源域选择模型(CMSS)\cite{yang2020curriculum}. 
MDAN,DCTN,MoE,MMN,MADAN和MDDA都需要源域样本的域标签. MDDA和CMSS利用
了静态权重机制选择距离目标域较近的源域样本,而其他模型并没有考虑不同源域样
本的重要性. 用于情感分类的的多源域适应方法,\textit{e.g.}MDAN和MoE仅仅集中于抽取域无
关特征,这将导致目标域中与情感相关的域特征被丢失. 区别于这些方法,对于源域
样本,我利用循环一致性和情感一致性生成了一个距离目标域较近的中间域. 更进一
步地,我提出实例级动态权重机制,在不需要域标签的情况下对源域样本分配权重. 
% !TeX root = ../main.tex
% -*- coding: utf-8 -*-

\chapter{常用包}
\label{chpt:method}

\section{The Tikz 绘图Package}
\label{sec:method:tikz}


The {\scshape pdf}\ package, where ``{\scshape pdf}'' is supposed to mean ``portable
graphics format'' (or ``pretty, good, functional'' if you
prefer\dots), is a package for creating graphics in an ``inline''
manner. It defines a number of \TeX\ commands that draw
graphics. For example, the code \verb|\tikz \draw (0pt,0pt) -- (20pt,6pt);|
yields the line \tikz \draw (0pt,0pt) -- (20pt,6pt); and the code \verb|\tikz \fill[orange] (1ex,1ex) circle (1ex);| yields \tikz
\fill[orange] (1ex,1ex) circle (1ex);.

\begin{figure}[h]
    \centering
    % !TeX root = ../main.tex

\begin{tikzpicture}[->,>=stealth,shorten >=1pt,auto,node distance=2.8cm,semithick]
  \tikzstyle{every state}=[fill=yellow,draw=none,text=black]

  \node[state]         (S) at (-6, 0)              {$S$};
  \node[state]         (xin1) at (-2, 3)           {$X^1_{in}$};
  \node[state]         (xin2) at (-2, 1)        {$X^2_{in}$};
  \node[state]         (xin3) at (-2, -1)       {$X^3_{in}$};
  \node[state]         (xin4) at (-2, -3)           {$X^4_{in}$};
  \node[state]         (xout1) at (0, 3)          {$X^1_{out}$};
  \node[state]         (xout2) at (0, 1)        {$X^2_{out}$};
  \node[state]         (xout3) at (0, -1)   {$X^3_{out}$};
  \node[state]         (xout4) at (0, -3)           {$X^4_{out}$};
  \node[state]         (xin5)  at (3, -2)   {$X^5_{in}$};
  \node[state]         (xout5) at (5, -2)   {$X^5_{out}$};
  \node[state]         (DC) at (7, 2)           {$DC$};

  \path (S) edge[bend left=26]              node {$\infty$} (xin1)
            edge[bend left=12]              node {$\infty$} (xin2)
            edge[bend right=12]             node {$\infty$} (xin3)
            edge[bend right=26]             node {$\infty$} (xin4)
        (xin1) edge  node {$\alpha=1$} (xout1)
        (xin2) edge  node {$\alpha=1$} (xout2)
        (xin3) edge  node {$\alpha=1$} (xout3)
        (xin4) edge  node {$\alpha=1$} (xout4)
        (xin5) edge  node {$1$} (xout5);
  \draw[->] (xout1) to[out=-30,in=150] node {$\beta$} (xin5);
  \draw[->] (xout2.east) to[out=-15,in=165] node [below] {$\beta$} (xin5);
  \draw[->] (xout3.east) to[out=0,in=180] node [below] {$\beta$} (xin5.west);
  \draw[->] (xout1) to[out=-5,in=175] node {$\infty$} (DC);
  \draw[->] (xout5) to[out=40, in=-120] node {$\infty$} (DC);
\end{tikzpicture}
 
    \caption{\label{fig:exmaple1} 示例图1}
\end{figure}

In a sense, when you use {\scshape pdf}\ you ``program'' your graphics, just
as you ``program'' your document when you use \TeX.  You get all
the advantages of the ``\TeX-approach to typesetting'' for your
graphics: quick creation of simple graphics, precise positioning, the
use of macros, often superior typography. You also inherit all the
disadvantages: steep learning curve, no \textsc{wysiwyg}, small
changes require a long recompilation time, and the code does not
really ``show'' how things will look like.





\begin{figure}
    \centering
    % !TeX root = ../main.tex

\begin{tikzpicture}[node distance=2cm]
 %定义流程图具体形状
 \tikzstyle{startstop} = [rectangle, rounded corners, minimum width=3cm, minimum height=1cm,text centered, draw=black, fill=red!30]
 \tikzstyle{io} = [trapezium, trapezium left angle=70, trapezium right angle=110, minimum width=3cm, minimum height=1cm, text centered, draw=black, fill=blue!30]
 \tikzstyle{process} = [rectangle, minimum width=3cm, minimum height=1cm, text centered, draw=black, fill=orange!30]
 \tikzstyle{decision} = [diamond, minimum width=3cm, minimum height=1cm, text centered, draw=black, fill=green!30]
 \tikzstyle{arrow} = [thick,->,>=stealth]
 
\node (start) [startstop] {Start};
\node (in1) [io, below of=start] {Input};
\node (pro1) [process, below of=in1] {Process 1};
\node (dec1) [decision, below of=pro1, yshift=-0.5cm] {Decision 1};
\node (pro2a) [process, below of=dec1, yshift=-0.5cm] {Process 2a};
\node (pro2b) [process, right of=dec1, xshift=2cm] {Process 2b};
\node (out1) [io, below of=pro2a] {Output};
\node (stop) [startstop, below of=out1] {Stop};
 
 %连接具体形状
\draw [arrow](start) -- (in1);
\draw [arrow](in1) -- (pro1);
\draw [arrow](pro1) -- (dec1);
\draw [arrow](dec1) -- (pro2a);
\draw [arrow](dec1) -- (pro2b);
\draw [arrow](dec1) -- node[anchor=east] {yes} (pro2a);
\draw [arrow](dec1) -- node[anchor=south] {no} (pro2b);
\draw [arrow](pro2b) |- (pro1);
\draw [arrow](pro2a) -- (out1);
\draw [arrow](out1) -- (stop);
\end{tikzpicture}

    \caption{\label{fig:exmaple2} 示例流程图2}
\end{figure}


\section{代码块}
\label{sec:method:code}

python 代码可以直接使用\textbf{python}环境

\begin{python}[caption={斐波那契Python}]
def fibonacci(n):
    # Fibonacci number
    if n < 0:
        return False
    if n <= 1:
        return n
    return fibonacci(n-2) + fibonacci(n-1)
\end{python}

C/C++ 代码可以直接使用\textbf{cpp}环境

\begin{cpp}[caption={斐波那契C++}]
unsigned long Fibonacci(int n)
{
    // Fibonacci start from 0
    if (n <= 1) 
    {
        return n;
    }
    else 
    {
        return Fibonacci(n - 1) + Fibonacci(n - 2);
    }
}
\end{cpp}

其他代码,使用\textbf{lstlisting}指明 \textbf{language}即可,如matlab代码

\begin{lstlisting}[caption={Matlab代码},language=Matlab]
function a = factorial(n)
% return n!
    if n==0
        a=1;
    else
        a=n * factorial(n-1);
    end
\end{lstlisting}
% !TeX root = ../main.tex
% -*- coding: utf-8 -*-

\chapter{讨论}



\section{\TeX\ 简介}

以下内容是 milksea@bbs.ctex.org 撰写的关于\TeX\ 的简单介绍。
注意这不是一个入门教程,不讲 \TeX\ 系统的配置安装,也不讲具体的 \LaTeX\ 代码。
这里仅仅试图以一些只言片语来解释:
进入这个门槛之前新手应该知道的注意事项,以及遇到问题以后该去如何解决问题。

\subsection{什么是 \TeX/\LaTeX,我是否应该选择它}

\TeX\ 是最早由高德纳(Donald Knuth)教授创建的一门标记式宏语言,
用来排版科技文章,尤其擅长处理复杂的数学公式。\TeX\ 同时也是处理这一语言的排版软件。
\LaTeX\ 是 Leslie Lamport 在 \TeX\ 基础上按内容/格式分离和模块化等思想建立的一集 \TeX\ 上的格式。

\TeX\ 本身的领域是专业排版(即方正书版、InDesign 的领域),
但现在 TeX/LaTeX 也被广泛用于生成电子文档甚至幻灯片等,\TeX\ 语言的数学部分
偶尔也在其他一些地方使用。但注意 \TeX\ 并不适用于文书处理(MS Office 的领域,以前和现在都不是)。

选择使用 \TeX/\LaTeX\ 的理由包括:
\begin{itemize}
\item 免费软件;
\item 专业的排版效果;
\item 是事实上的专业数学排版标准;
\item 广泛的西文期刊接收甚或只接收 LaTeX 格式的投稿;
\item[] ……
\end{itemize}
不选择使用 \TeX/\LaTeX\ 的理由包括:
\begin{itemize}
\item 需要相当精力学习;
\item 图文混合排版能力弱;
\item 仅流行于数学、物理、计算机等领域;
\item 中文期刊的支持较差;
\item[] ……
\end{itemize}

请尽量清醒看待网上经常见到的关于 \TeX\ 与其他软件的优劣比较和口水战。在选择使用或离开之前,请先考虑
\TeX\ 的应用领域,想想它是否适合你的需要。

\def\AAAA{}

\subsection{我该用什么编辑器?}

编辑器功能有简有繁,特色不一,从简单的纯文本编辑器到繁复的 Emacs,因人而易。基本功能有语法高亮、方便编译预览就很好了,扩充功能和定制有无限的可能。初学者可以使用功能简单、使用方便的专用编辑器,如 TeXWorks、Kile、WinEdt 等,或者类似所见即所得功能的 LyX;熟悉的人可以使用定制性更强的 Notepad++、SciTE、Vim、Emacs 等。这方面的介绍很多,一开始不妨多试几种,找到最适合自己的才是最好的。

另外提醒一句,编辑器只是工作的助力,不必把它看得太重。一些编辑器有极为繁杂的功能,一些编辑器常常会引来黑客们的论战(如 Emacs 与 Vim)。为工作,别为这些浪费太多精力,适用即可。

\subsection{我该去哪里寻找答案?}

0、绝对的新手,先读完一本入门读物,了解基本的知识。

1、无论如何,先读文档!绝大部分问题都是文档可以解决的。

2、再利用 Google 搜索,利用(bbs.ctex.org)版面搜索。

3、清楚、聪明地提出你的问题。


\subsection{我应该看什么 \LaTeX\ 读物?}

这不是一个容易回答的问题,因为有许多选择,也同样有许多不合适的选择。
这里只是选出一个比较好的答案。更多更详细的介绍可以在版面和网上寻找(注意时效)。

近两年 \TeX\ 的中文处理发展很快,目前没有哪本书在中文处理方面给出一个最新进展的合适综述,
因而下面的介绍也不主要考虑中文处理。

\begin{enumerate}
\item 我可以阅读英文

\begin{enumerate}
\item 我要迅速入门:ltxprimer.pdf (LaTeX Tutorials: A Primer, India TUG)
\item 我要系统学习:A Guide to LaTeX, 4th Edition, Addison-Wesley
      有机械工业出版社的影印版(《LaTeX实用教程》)
\item 我要深入学习:要读许多书和文档,TeXbook 是必读的
\item 还有呢?去读你使用的每一个宏包的说明文档
\item 还有许多专题文档,如讲数学公式、图形、表格、字体等
\end{enumerate}

\item 我更愿意阅读中文
\begin{enumerate}
\item 我要迅速入门:lnotes.pdf (LaTeX Notes, 1.20, Alpha Huang)
\item 我要系统学习:《LaTeX2ε 科技排版指南》,邓建松(电子版)
 如果不好找,看《LaTeX 入门与提高》第二版,陈志杰等
\item 我要深入学习:TeXbook0.pdf (特可爱原本,TeXbook 的中译,xianxian)
\item 还有呢?英语,绝大多数 TeX 资料还是英文的
\end{enumerate}
\end{enumerate}

\subsection{什么知识会过时?什么不会?}

\TeX\ 是排版语言,也是广泛使用的软件,并且不断在发展中;
因此,总有一些东西会很快过时。作为学习 \TeX\ 的人,
免不了要看各种各样的书籍、电子文档和网络论坛上的只言片语,
因此了解什么知识会迅速过时,什么知识不会是十分重要的。

最稳定的是关于 Primitive \TeX\ 和 Plain \TeX\ 的知识,也就是 Knuth
在他的《The TeXbook 》中介绍的内容。因为 \TeX\
系统开发的初衷就是稳定性,要求今天的文档到很久以后仍可以得到完全相同的结果,
因此 Knuth 限定了他的 \TeX\ 语言和相关实现的命令、语法。这些内容许多年来就没有多少变化,
在未来的一些年里也不会有什么变化。
Primitive \TeX\ 和 Plain \TeX\ 的知识主要包括 \TeX\ 排版的基本算法和原理,
盒子的原理,底层的 \TeX\ 命令等。其中技巧性的东西大多在宏包设计中,
初学者一般不会接触到很多;而基本原理则是常常被提到的,
譬如,\TeX\ 把一切排版内容作为盒子(box)处理。

相对稳定的是关于基本 \LaTeXe\
的知识,也包括围绕 \LaTeXe\ 的一些核心宏包的知识。\LaTeXe\
是自 1993 年以来的一个稳定的 \LaTeX\ 版本,直到最近的一次修订
(2005 年)都没有大的变动。
\LaTeX\ 的下一个计划中的版本 \LaTeX 3 遥遥无期,在可预见的将来,\LaTeXe\ 不会过时。
\LaTeXe\ 的知识是目前大部分 \LaTeX\ 书籍的主体内容。关于 \LaTeX\ 的标准文档类
(article、report、book、letter、slide 等),关于基本数学公式的输入,
文档的章节层次,表格和矩阵,图表浮动体,LR 盒子与段落盒子……
这些 \LaTeX\ 的核心内容都是最常用的,相对稳定的。
与 \LaTeXe\ 相匹配的核心宏包,
如 graphics(x)、ifthen、fontenc、doc 等,也同样是相对稳定的。
还有一些被非常广泛应用的宏包,如 amsmath 系列,也可以看作是相对稳定的。

简单地说,关于基本 \TeX/\LaTeX\ 的语言,都是比较稳定的。与之对应,实现或者支持 \TeX/\LaTeX\ 语言的软件,
包括在 \TeX/\LaTeX\ 基础上建立的新的宏,都不大稳定。

容易过时的是关于第三方 \LaTeX\ 宏包的知识、第三方 \TeX\ 工具的知识,以及新兴 \TeX\ 相关软件的知识等。
\TeX\ 和 \LaTeX\ 语言是追求稳定的;但无论是宏包还是工具,作为不断更新软件,它们是不稳定的。
容易过时的技术很多,而且现在广泛地出现在几乎所有 \LaTeX\ 文档之中,因此需要特别引起注意:
宏包的过时的原因可能是宏包本身的升级换代带来了新功能或不兼容,
也可能是同一功能的更新更好的宏包代替了旧的宏包。前者的典型例子比如绘图宏包 PGF/TikZ,
现在的 2.00 版功能十分强大,和旧的 1.1x 版相差很大,和更旧的 0.x 版本则几乎完全不同;后
者的典型例子比如 caption 宏包先是被更新的 caption2 宏包代替,后来 caption 宏包更新又使得
caption2 宏包完全过时。——安装更新的发行版可以避免使用过旧的宏包;
认真阅读宏包自带的文档而不是搜索得到的陈旧片断可以避免采用过时的代码。

工具过时的主要原因也是升级换代和被其他工具替换。前者的典型例子是编辑器
WinEdt 在 5.5 以后的版本支持 UTF-8 编码,而旧版本不支持;
后者的典型例子是中文字体安装工具从 GBKFonts 到 xGBKFonts 到 FontsGen 不断被取代。
图形插入是一个在 \TeX\ 实现、宏包与外围工具方面都更新很快的东西。
在过去,最常用的输出格式是 PS(PostScript)格式,因此插入的图像以 EPS 为主流。
使用 Dvips 为主要输出工具,外围工具有 GhostScript、bmeps 等等,相关宏包有 graphics 等,
相关文档如《LaTeX2e 插图指南》。

但凡提及“\LaTeX\ 只支持 EPS 图形”的,就是这个过时的时代的产物。事实上 \TeX/\LaTeX\
并不限定任何图形格式,只不过是当时的输出格式(PS)和工具(Dvips)对 EPS 情有独钟而已。
后来 PDF 格式成为主流,pdf\TeX、DVIPDFM、DVIPDFMx、\XeTeX\ 等工具则主要支持 PDF、PNG、JPG 格式的图形,
涉及一系列工具如 ImageMagick、ebb 等。

值得特别提出注意的就是,中文处理也一起是更新迅速、容易过时的部分。
而且因为中文处理一直没有一个“官方”的“标准”做法,软件、工具、
文档以及网上纷繁的笔记也就显得相当混乱。从八十年代开始的 CCT 系统、
天元系统,到后来的 CJK 方式,到近来的 \XeTeX、LuaTeX 方式,
中文处理的原理、软件、宏包、配置方式等都在不断变化中。



\subsection{插图格式}

前面提到, \LaTeX\ 主要支持EPS格式的插图文件, 而PDF\LaTeX\ 则更喜欢 PDF、PNG、JPG 格式的图形。
为解决兼容性,最新版的 PDF\LaTeX 会自动把 EPS 文件转换为 PDF 文件。因此,使用 EPS 格式的插图可能具有最广泛
的通用性。

\subsection{\LaTeX\ 作图}

目前已经有很多优秀的\LaTeX\ 作图宏包,如 pgf/Tikz 和 pstricks,两者都具有强大的作图能力。
% !TeX root = ../main.tex
% -*- coding: utf-8 -*-
\chapter{总结展望}

任何问题可在\href{https://github.com/kongxiao0532/NKU\_Bachelor\_Thesis\_Template/issues/new}{Github Repo}上发起issue

% !TeX root = ../main.tex
% -*- coding: utf-8 -*-

\printbibliography

% !TeX root = ../main.tex
% -*- coding: utf-8 -*-

%\makeschapterhead{致谢}
\chapter*{致谢}
感谢您使用本模板。

% !TeX root = ../main.tex
% -*- coding: utf-8 -*-

\end{document}
