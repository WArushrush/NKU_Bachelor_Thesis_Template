% !TEX encoding = UTF-8 Unicode
% !TEX program = xelatex
% !BIB program = biber
% !TEX TS-program = xelatex
% !BIB TS-program = biber
%%
%%  本模板方式编译: XeLaTeX + biber
%%
%%  注意: 在改变编译方式前应先删除 *.toc 和 *.aux 文件
%%
\documentclass[12pt,openright]{book}

% 引入NKThesis包
\usepackage[emptydoublepage]{NKThesis}   % 中文
%\usepackage[emptydoublepage,English]{NKThesis} % 英文

% 其它包按需添加
% \usepackage{amsmath}
% \usepackage{cases}
% \usepackage{multirow}
% \usepackage{tocbibind}
\usepackage{ulem}
\usepackage{xeCJKfntef} %CJKunderline
\xeCJKsetup{underline= {format = \color{black}, thickness=0.4pt}} %CJKunderline setting

% 参考文献
\addbibresource{nkthesis.bib}
% 图片文件夹
\graphicspath{{image/}}

\includeonly{
	./tex/abstract,
	./tex/introduction,
	./tex/relatedwork,
	./tex/method,
	./tex/discussion,
	./tex/summary,
	./tex/references,
	./tex/acknowledgements,
	./tex/appendices,
	./tex/resume
}
\begin{document}
%  设置基本信息
%  注意:  逗号`,'是项目分隔符. 如果某一项的值出现逗号, 应放在花括号内, 如 {,}
%
\NKTsetup{
	% 封面设置
	论文题目(中文) = 基于对抗网络的多源域适应,
	论文题目(英文) = Adversarial Network for Multiple Sources Domain Adaptation,
	学号           = 1711388,
	姓名       = 肖阳,
	年级          = 2017级,
	学院          = 计算机学院,
	系别          = 计算机系,
	专业          = 计算机科学与技术,
	完成日期      = ,
	指导教师       = 杨征路教授,
	校外指导教师   = 赵思成博士{,}郭江博士{,}岳翔宇博士{,}宋晓林博士,
}


% !TeX root = ../main.tex
% -*- coding: utf-8 -*-


\begin{zhaiyao}
    对于社交网络中用户对商品和服务的评论进行情感分析可以有效地帮助企业或公司量化分析客户给予的反馈并做出改进。 为了减轻目标域上所需要的大量标注数据,领域自适应通过学习一个在源域上的可迁移模型提供了一个有效替代方案。目前已有的多源域适应(MDA)方法大多基于抽取域无关信息以最小化源域分布与目标域分布之间的距离, 因而无法保留目标域上特有的与情感相关的域特征,忽视了不同源域甚至同一源域的子域之间的相互关系, 并且无法依据模型状态准确地量化不同训练阶段每个样本的权重变化。在这篇论文中,我提出了一个新的实例级多源域适应框架,命名为课程式循环一致性对抗神经网络(C-CycleGAN)。具体地,C-CycleGAN包含三个部分:(1) 预训练的文本编码器,用来将不同领域的文本编码到一个连续的表征空间,(2) 课程式实例级领域自适应的中间域生成器,用来最小化源域与目标域间的域间隙,(3) 在中间域上训练的目标任务分类器,用来做最终的情感分类。C-CycleGAN将源域中的实例级样本迁移到一个距离目标域更近的中间域上,同时保留尽可能多的情感语义信息并丢失尽可能少的域特征。进一步地,模型中的动态实例级权重机制能够实时地在每个训练阶段给不同的源域样本分配最优权重。我在三个基准数据集上做了大量实验并相较于最前沿方案取得了大幅提升。
\end{zhaiyao}




\begin{guanjianci}
    迁移学习,领域自适应,循环生成对抗网络, 课程式学习,多源域,情感分析,自然语言处理
\end{guanjianci}



\begin{abstract}
    Sentiment analysis of user-generated reviews or comments on products and services in social networks can help enterprises to analyze
    the feedback from customers and take corresponding actions for
    improvement. To mitigate large-scale annotations on the target
    domain, domain adaptation (DA) provides an alternate solution
    by learning a transferable model from other labeled source domains. Existing multi-source domain adaptation (MDA) methods
    either fail to extract some discriminative features in the target
    domain that are related to sentiment, neglect the correlations of
    different sources and the distribution difference among different
    sub-domains even in the same source, or cannot reflect the varying
    optimal weighting during different training stages. In this paper, we
    propose a novel instance-level MDA framework, named curriculum
    cycle-consistent generative adversarial network (C-CycleGAN), to
    address the above issues. Specifically, C-CycleGAN consists of three
    components: (1) pre-trained text encoder which encodes textual input from different domains into a continuous representation space,
    (2) intermediate domain generator with curriculum instance-level
    adaptation which bridges the gap across source and target domains,
    and (3) task classifier trained on the intermediate domain for final sentiment classification. C-CycleGAN transfers source samples
    at instance-level to an intermediate domain that is closer to the
    target domain with sentiment semantics preserved and without
    losing discriminative features. Further, our dynamic instance-level
    weighting mechanisms can assign the optimal weights to different 
    source samples in each training stage. We conduct extensive experiments on three benchmark datasets and achieve substantial gains
    over state-of-the-art DA approaches.
\end{abstract}



\begin{keywords}
    Domain adaptation, multiple sources, sentiment analysis, cycle-consistent generative adversarial network, curriculum learning
\end{keywords} 

\tableofcontents

% !TeX root = ../main.tex
% -*- coding: utf-8 -*-
% !TeX root = ../main.tex
% -*- coding: utf-8 -*-

\chapter{绪论}
\label{chpt:introduction}


广泛流行的社交网络和移动设备使人类能够对他们关于线上购买的商品和服务运用文本,图片和视频等方式评价和分享他们的观点和想法\cite{zhao2015predicting,zhao2016predicting,deriu2017leveraging,gong2017clustered,luiz2018feature,gong2018sentiment,zhao2020discrete,zhao2020end}.例如,当计划购物时,很大概率我们会看其他人对于这个商品或服务的评价.如果用户的反馈是以负面评价主导的,我们也许会转而考虑其他的品牌.对于用户生成的大规模多媒体数据进行情感分析不仅可以帮助顾客选择他们需要的物品并且还可以促使企业提高他们产品和服务的质量\cite{zhao2016predicting,chen2019emoji}.最近的研究\cite{zhang2018textual,kiritchenko2014sentiment,georgakopoulos2018convolutional,liu2018content,yadav2020sentiment,wang2018sentiment,wang2019aspect,chen2019emoji,biddle2020leveraging}已经证明了深度神经网络(DNN)在情感分析中已经取得了最前沿的表现.但是,训练一个深度神经网络并使其达到它的能力上限通常需要很大规模的标注数据,这需要大量的时间和人力资源才能得到.一个方案是在一个有标注数据的源域上面训练并在目标域上直接使用.但是由于域间隙的存在\cite{torralba2011unbiased},\textit{i.e.} 源域和目标域的分布不同,这种方案将会导致神经网络在目标域上的表现明显退化\cite{tzeng2015simultaneous,hoffman2018cycada,yue2019domain,yang2020curriculum}.最小化域间隙的领域自适应(DA)\cite{patel2015visual,sun2015survey,kouw2019review,zhao2020multi,zhao2020review}方法提供了一个替代方案:在源域上训练一个对于目标域泛化能力强的模型.
目前的领域自适应方法主要集中于单源无监督学习\cite{liu2019survey,xi2020domain},\textit{i.e.} 只有一个有标注过的源域和一个无标注的目标域.然而这些无监督领域自适应(UDA)方法当且仅当源域与目标域的域间隙相对较小时才会表现很好,在实际场景下,域间隙很大或源域的个数大于一时,它们的表现会严重退化\cite{guo2018multi,zhao2020multi}.例如,如果我们有一个目标域:厨房,其中包含用户对于碗,菜谱,平底锅和电水壶等评论,以及三个源域:书籍,电器和电影, 完美地将每个源域与目标域对齐是难以实现的.一个直观的解决方案是将三个源域的数据组合在一起成为一个源域,之后通过单源无监督领域自适应的方法对齐源域与目标域.然而这样的方法只会得到次优结果,如图1.充分探索不同源域间的复杂相关信息能够在目标域上得到更好的结果,这催生了多源域适应(MDA)方法\cite{sun2015survey,zhao2020multi}.
最近有一些深度多源域适应方法被提出,其中大部分基于对抗学习,其中包含一对特征提取器和域判别器,用来最小化特征空间中源域分布与目标域分布的JS散度(\textit{e.g.}:MDAN\cite{zhao2018adversarial},MOE\cite{guo2018multi}),还有一部分基于最小化分布差异,其中包含一个特征提取器和一个核函数,用来最小化再生希尔伯特空间中源域和目标域的分布差异损失函数(例如:MMD损失\cite{dziugaite2015training},CDD损失\cite{Kang_2019_CVPR}).这些方法主要集中于抽取不同领域的域无关特征,将每个源域与目标域分别对齐,或从统计意义上对给源域样本设置权重.这些方法虽然可以得到不同领域的域无关特征,但仍然有一些局限.首先,在特征抽取的过程中一些目标域上的与情感相关的域特征被丢失,这是因为域间共享的特征提取器旨在通过将源域和目标域样本投影到一个低维空间中来提取域无关特征,因此不会包含目标域中全部的情感相关特征,同时可以理解为这是一个压缩的不可逆过程,易导致模式崩塌\cite{zhu2017unpaired,zhao2019cycleemotiongan,hoffman2018cycada}.其次,一些现存的多源域适应方法利用域标签将源域和目标域分别对齐,这忽视了不同源域甚至同一领域的不同子域间的相互关系.这些方法在域标签不存在的情况下将会自然地退化为单源域适应并得到次优解.最后,现存的基于采样的方法集中于通过训练样本选择模型来筛选距离目标域较近的源于样本(\textit{e.g.}:MDDA\cite{zhao2020distilling},CMSS\cite{yang2020curriculum}),这并不能即时并准确地在不同的训练阶段量化样本的动态最优权重,这是因为在不同的训练阶段,模型对同一个样本的学习进度是不同的,而一个收敛的,以样本为输入,以权重为输出的样本选择模型需要在不同的训练阶段对同一个样本输入去输出不同的权重,这与收敛的定义相悖.
在这篇文章中,我提出了一个新的实例级多源域适应框架,命名为课程式循环一致性生成对抗网络(C-CycleGAN),来解决上述问题.首先,为了将源域和目标域中样本的全部信息映射到一个连续平滑的表征空间并最小化信息损失,我引入了重构方法以更好地保留信息.其次,对于编码后的源域样本表征,我利用循环一致性生成对抗网络(CycleGAN)生成了一个中间域以将混合后的源域和目标域进行对齐.为了量化一个batch中不同源域样本对于模型的重要性,我通过动态的基于模型和无模型的两种权重机制对实例级样本分配权重.情感损失函数同样会反向传播给源域到目标域的生成器来保留情感信息.我在三个基准数据集上做了大量实验:Reviews-5\cite{yu2016learning},Amazon benchmark\cite{chen2012marginalized},和Multilingual Amazon Reviews Corpus\cite{chen2019multi}.结果表明提出的C-CycleGAN模型超越了情感分类任务领域自适应的SOTA方法.
综上所述, 本片文章的贡献分为三部分:
(1)我提出了一个新的多源域适应方法,命名为课程式循环一致性生成对抗网络(C-CycleGAN)以最小化多个源域与目标域之间的域间隙.
(2)我设计了新的实例级基于模型和无模型两种权重机制,可以动态更新样本权重.利用这种方法,C-CycleGAN不需要任何域标签,并且可以更好地挖掘多源域间的复杂相互关系.实验表明利用这种方法得到的结果优于利用域标签.
(3)我在三个基准数据集上做了大量实验.与最佳的基线模型相比,C-CycleGAN在平均分类准确率这一指标上在Reviews-5,Amazon benchmark和Multilingual Amazon Reviews Corpus数据集上分别提升了1.6\%,1.2\%和13.4\%.
% !TeX root = ../main.tex
% -*- coding: utf-8 -*-


\chapter{相关工作} 
\label{chpt:relatedwork}

\textbf{文本情感分类} 文本情感分类或意见挖掘旨在从文本方面评估人们对于商品,
服务或组织的观点,情感和态度\cite{zhang2018deep}. 诸如商品评论,论坛讨论和微信等广泛流
行的社交网络促成了这一任务的迅速发展\cite{zhang2018deep,chen2019emoji}. 传统的情感分析方法主
要专注于设计人工特征\cite{pang2008opinion,mohammad2013nrc},将其送入标准的分类器,例如SVM. 近年
情感分析的工作主要基于深度神经网络\cite{zhang2018deep,wang2018sentiment},其已经胜任了许多自然语
言处理任务. 一些广泛应用于情感分析的深度模型包括递归自编码器\cite{socher2011semi,dong2014adaptive,qian2015learning},
递归神经张量网络\cite{socher2013recursive},循环神经网络(RNN)\cite{cho2014learning},长短期记忆神经网络(LSTM)\cite{hochreiter1997long},
树状LSTM\cite{tai2015improved},RNN编码解码结构\cite{cho2014learning}和BERT\cite{devlin2019bert}. 上述监督学习方法通常需要大
规模的标注数据进行训练. 然而高质量的情感标签通常需要消耗大量的时间和人力
资源才能得到. 在C-CycleGAN中,我应用了双向LSTM(Bi-LSTM)\cite{hochreiter1997long}作为编码器
和一个多层感知机作为最终情感分类任务的分类器. \\
\textbf{单源无监督域适应} 近年单源无监督域适应(SUDA)方法大多主要采用包含两条尾部相
连的数据流的深度神经网络\cite{zhuo2017deep,zhao2018adversarial}. 一条数据流在带标记数据的源域上用传统的
目标任务损失函数进行训练,例如分类任务的交叉熵损失(cross entropy loss). 
另一条数据流旨在通过不同的对齐损失函数(alignment loss)将源域与目标域进行
对齐以解决域间隙问题,例如分布差异损失(discrepancy loss),对抗损失(adversarial loss)
和自监督损失(self-supervised loss)\emph{etc}. 基于分布差异的方法采用一些距离度
量来最小化源域和目标域在特定空间的差异,例如最大期望差异\cite{long2015learning,wang2018multi,xi2020domain},相
关性对齐\cite{sun2016return,sun2017correlation,zhuo2017deep}和对比领域差异\cite{kang2019contrastive}. 对比判别模型
(adversarial discriminative model)通常采用一个域判别器以对抗地对齐从
源域和目标域中抽取的特征来使他们无法判别\cite{ganin2016domain,tzeng2017adversarial,chen2017no,shen2017wasserstein,tsai2018learning,huang2018domain,kumar2019adversarial,wu2020unsupervised}. 
除了域判别器,对抗生成模型(adversarial generative model)还包含一个生成部分
,基于生成对抗网络(GAN)\cite{goodfellow2014generative}和它的变体,例如CoGAN\cite{liu2016coupled}, SimGAN\cite{shrivastava2017learning}和
CycleGAN\cite{zhu2017unpaired,zhao2019cycleemotiongan,hoffman2018cycada}以生成假源域或目标域样本. 基于自监督的模型在原始的目
标任务网络中融入了辅助的自监督学习任务以使源域和目标域靠近. 通常使用的自监
督任务包括重构(reconstruction)\cite{ghifary2015domain,ghifary2016deep,chen2020fido},图像旋转预测(image rotation prediction)\cite{sun2019unsupervised,xu2019self}
,拼图预测(jigsaw prediction)\cite{carlucci2019domain},和掩码(masking)\cite{vu2020effective}. 尽管这些方法在
SUDA任务中达到了满意的效果,他们的表现在应用于多源域适应任务时会大幅衰减. \\
\textbf{多源域适应} 基于一些理论分析\cite{ben2010theory,hoffman2018algorithms},多源域适应旨在更好地解决训练数据属于
多个源域的问题\cite{sun2015survey,zhao2019multi}. 早期的多源域适应方法主要分为两种\cite{sun2015survey,sun2011two,duan2012exploiting,chattopadhyay2012multisource,duan2012domain}以及预训练分类器的组合\cite{xu2012multi,sun2013bayesian}. 近年的方法考虑了一些特殊的
多源域适应场景,例如不完全多源域适应\cite{ding2018incomplete}和目标域偏移\cite{redko2019optimal}. 
最近一些基于深度表示学习的多源域适应方法被提出,例如多源对抗网络(MDAN)\cite{zhao2018adversarial},
深度混合网络(DCTN)\cite{xu2018deep},混合专家模型(MoE)\cite{guo2018multi},时矩匹配网络(MMN)\cite{peng2019moment},
多源对抗聚合网络(MADAN)\cite{zhao2019multi},多源蒸馏模型(MDDA)\cite{zhao2020distilling}和课程式源域选择模型(CMSS)\cite{yang2020curriculum}. 
MDAN,DCTN,MoE,MMN,MADAN和MDDA都需要源域样本的域标签. MDDA和CMSS利用
了静态权重机制选择距离目标域较近的源域样本,而其他模型并没有考虑不同源域样
本的重要性. 用于情感分类的的多源域适应方法,\textit{e.g.}MDAN和MoE仅仅集中于抽取域无
关特征,这将导致目标域中与情感相关的域特征被丢失. 区别于这些方法,对于源域
样本,我利用循环一致性和情感一致性生成了一个距离目标域较近的中间域. 更进一
步地,我提出实例级动态权重机制,在不需要域标签的情况下对源域样本分配权重. 
\include{./tex/method}
\include{./tex/discussion}
\include{./tex/summary}

\include{./tex/references}
\include{./tex/acknowledgements}
\include{./tex/appendices}

\end{document}
